%%% File encoding is ISO-8859-1 (also known as Latin-1)
%%% You can use special characters just like �,� and �

% Chapter without numbering but with appearance in the Table of Contents
% \addchap is a command from KOMA-Script
\addchap{Executive Summary}
Search pervades everyday life in the twenty-first century. Students, researchers, and ordinary people alike enjoy the privilege of record-speed information retrieval from the Internet. Despite its fundamental importance, search is complicated. It leans heavily on the subject of text mining---the extraction of pertinent information from natural language documents---a subject which faces many challenges, both theoretical and practical. How does one model the content of a textual document? The same words appear in many documents, but each document may use those words differently. What determines a document's semantics? What about semantic similarity? Most importantly, how can one measure the semantic features of a document in real time? It turns out that many of these questions can be addressed at least in part by machine learning. In particular, the issue of semantic categorization can be reduced to a multiple-classification problem, one that is accurately solved by support vector machines.
\\\\
We address these questions in the present paper, namely the questions of textual document representation and categorization. We will discuss methods for representing textual data, isolating the semantics of a textual document, and predicting attributes of new documents with learned classifiers. Specifically, we will discuss: term-frequency/inverse-document-frequency representations of text and measures of similarity between different texts; semantic noise reduction of many documents via latent semantic analysis; and supervised categorization of textual documents using k-nearest-neighbors and support vector machines. At the end, the reader will be tasked with building a simple search engine for the Reuters-21578 corpus, whose implementation will depend on the methods of text mining discussed in the background. 